\chapter*{Kurzfassung}


In der vorliegenden Diplomarbeit geht es um die Entwicklung eines Systems, wo die wichtigsten Ank\"{u}ndigungen des Tages auf einem Bildschirm dargestellt werden. Auf diese Weise wird eine Innovation im Erhalt der Informationen f\"{u}r die Schule erreicht.

Deswegen war es notwendig, ein System zu erstellen, wo alle diese Informationen in einer digitalisierten Form angezeigt werden. Dieses System hei\ss{}t Infotainment-System. Der erstellte Prototyp besteht aus einem Raspberry PI Client und einem Raspberry PI Server. Im Raspberry PI Client befindet sich eine Website, die mit dem Server kommuniziert. Der Administrator kann sich bei der Webseite einloggen und um dort die verschiedenen Funktionalit\"{a}ten zu verwalten. Er erh\"{a}lt die M\"{o}glichkeit, die Informationen auszuw\"{a}hlen, die er auf dem Bildschirm anzeigen lassen will, sowie auch das entsprechende Layout. 

Die Informationen werden direkt aus einer von uns erstellten Datenbank ausgew\"{a}hlt, die schulrelevante Informationen wie Unterrichtsplan, Lehrerplan usw. enth\"{a}lt. Am Bildschirm k\"{o}nnen auch noch Kalenderinformationen (Termine, Olympiade), Wetterdaten und letzter Post von der Webseite der Schule dargestellt werden. Die Wetterdaten und den letzten Post bekommen wir mithilfe von APIs.  
Unser System kann mehrere Bildschirme mit unterschiedlichen Inhalten haben. Alle Bildschirme k\"{o}nnen auf relativ einfache und effiziente Weise von der Webseite aus verwaltet werden.   

Das Infotainment-System ist auch zu Unterhaltungszwecken zu verwenden. \"{U}ber Chatbot haben die Sch\"{u}ler die M\"{o}glichkeit, verschiedene Fotos zu posten, die auf dem Bildschirm angezeigt werden. Dies k\"{o}nnen Bilder des Alltags sein, die den Schultag auf die eine oder andere Weise interessanter machen. 
