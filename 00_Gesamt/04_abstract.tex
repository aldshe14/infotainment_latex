\chapter*{Abstract}


This diploma thesis deals with the development of a system where the most important announcements of the day are displayed on a screen. In this way, an innovation in obtaining information for the school is achieved. The previous method of writing information on paper was very tiresome and time consuming. Every day, new papers were printed, which were then hung on the information board.  

That was not an appealing task to the students. Therefore, it was necessary to create a system where all this information is displayed in a digitized form. This system is called infotainment system and consists of a Raspberry PI Client and a Raspberry PI Server. The Raspberry PI Client has a Website that communicates with the Server. The administrator can log in to the Website where they have the opportunity to select the information they want to display on the screen, as well as the corresponding layout. The information is selected directly from a database created by our team that contains school-related information such as lesson plan, teacher plan, and so on.  

Calendar information (dates, Olympics), weather data and the latest posts from the school's Website can also be displayed on the screen. Weather data and the latest posts can be received using APIs.  

Our system can have multiple screens with different content. All screens can be managed in a relatively simple and efficient way from the Website. The infotainment system can also be used for entertainment purposes. Chatbot allows students to post various photos which are going to be displayed on screen. The photographs will depict the daily school life of a Peter Mahringer student.

\newpage
\section*{P\"{e}rmbledhje}

Ideja e k\"{e}tij projekti \"{e}sht\"{e} realizimi i nj\"{e} sistemi, q\"{e} ka si funksion paraqitjen e njoftimeve m\"{e} t\"{e} r\"{e}nd\"{e}sishme t\"{e} dit\"{e}s n\"{e} nj\"{e} ekran. N\"{e} k\"{e}t\"{e} m\"{e}nyr\"{e}, do t\"{e} arrihet nj\"{e} risi n\"{e} m\"{e}nyr\"{e}n e transmetimit s\"{e} informacioneve p\"{e}rkat\"{e}se p\"{e}r shkoll\"{e}n. 

Metoda e m\"{e}parshme e p\"{e}rcjelljes s\"{e} informacionit ka qen\"{e} tep\"{e}r e lodhshme dhe k\"{e}rkonte nj\"{e} koh\"{e} t\"{e} konsiderueshme. Prandaj, ka qen\"{e} i nevojsh\"{e}m krijimi i nj\"{e} sistemi ku t\"{e} gjitha informacionet t\"{e} paraqiten n\"{e} nj\"{e} form\"{e} t\"{e} dixhitalizuar.  Prototipi q\"{e} ne kemi krijuar p\"{e}rb\"{e}het nga dy minikompjutera Raspberry PI, ku nj\"{e}ri kryen funksionin e klientit dhe tjetri at\"{e} t\"{e} serverit. Tek klienti ndodhet nj\"{e} uebit q\"{e} komunikon n\"{e} m\"{e}nyr\"{e} direkte me serverin. Administratori mund t\"{e} logohet aty dhe t\"{e} gjej\"{e} funksionalitete t\"{e} ndryshme q\"{e} i disponohen. Ai ka mund\"{e}sin\"{e} q\"{e} t\"{e} zgjedh\"{e} informacionet q\"{e} d\"{e}shiron t\"{e} paraqes\"{e} n\"{e} ekran, duke i selektuar ato direkt nga nj\"{e} baz\"{e} t\"{e} dh\"{e}nash. Kjo baz\"{e} e dh\"{e}nash \"{e}sht\"{e} krijuar nga ne dhe mbart informacionet relevante p\"{e}r shkoll\"{e}n. Administratori mund t\"{e} zgjedh\"{e} edhe nj\"{e} struktur\"{e} p\"{e}rkat\"{e}se p\"{e}r m\"{e}nyr\"{e}n e paraqitjes s\"{e} k\"{e}tyre t\"{e} dh\"{e}nave. 

N\"{e} ekran do t\"{e} tregohen edhe njoftime t\"{e} r\"{e}nd\"{e}sishme si p\"{e}r shembull datat e olimpiadave apo pushimeve.  Nj\"{e} funksionalitet i m\"{e}tejsh\"{e}m konsiston n\"{e} paraqitjen e t\"{e} dh\"{e}nave p\"{e}r motin apo edhe posti i fundit i uebit t\"{e} shkoll\"{e}s. Kjo arrihet me ndihm\"{e} t\"{e} API-ve. 
Nj\"{e} tjet\"{e}r tipar inovativ \"{e}sht\"{e} fakti q\"{e} sistemi yn\"{e} mund t\"{e} p\"{e}rdoret edhe p\"{e}r q\"{e}llime arg\"{e}tuese. N\"{e}p\"{e}rmjet Chatbotit, q\"{e} \"{e}sht\"{e} nj\"{e} komponent i inteligjenc\"{e}s artificiale, student\"{e}t mund t\"{e} postojn\"{e} vet\"{e} foto t\"{e} ndryshme n\"{e} ekran. K\"{e}to mund t\"{e} jen\"{e} fotografi t\"{e} jet\"{e}s s\"{e} p\"{e}rditshme, q\"{e} n\"{e} nj\"{e} m\"{e}nyre apo n\"{e} nj\"{e} tjet\"{e}r thyejn\"{e} monotonin\"{e} gjat\"{e} rutin\"{e}s ditore shkollore.