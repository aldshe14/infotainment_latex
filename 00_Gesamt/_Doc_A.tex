
\newcommand{\datitle}{Titel der Diplomarbeit}
\newcommand{\daort}{Shkoder}
\newcommand{\dabetreuer}{Titel Betreuer}
\newcommand{\dadate}{\today}
\newcommand{\daschoolyear}{2019/20}
\newcommand{\danumber}{17.xx}
\newcommand{\daclass}{5X}
\newcommand{\daschuelereins}{SchülerName1} %Nur wenn der Name verwendet wird Ändern -> bei der Eidesstattlichen Erkärung werden automatisch die nötigen Zeilen eingeblendet !
\newcommand{\daschuelerzwei}{SchülerName2}
\newcommand{\daschuelerdrei}{SchülerName3}
\newcommand{\daschuelervier}{SchülerName4}
\newcommand{\daschuelerfuenf}{SchülerName5}
\newcommand{\dabetreuereins}{Lehrer1}
\newcommand{\dabetreuerzwei}{Lehrer2}
\newcommand{\dabetreuerdrei}{Lehrer3}














%Base ordner for all other ordners (figures, appendix, ....)
\ifdefined\bordner
\renewcommand{\bordner}{./../00_Gesamt/}
\else
\newcommand{\bordner}{./../00_Gesamt/}
\fi


%Check Ordner for Figures
\ifdefined\ordnerfigures
\renewcommand{\ordnerfigures}{\bordner figures/}
\else
\newcommand{\ordnerfigures}{\bordner figures/}
\fi

%Check Ordner for Appendix
\ifdefined\ordnerappendix
\renewcommand{\ordnerappendix}{\bordner appendix/}
\else
\newcommand{\ordnerappendix}{\bordner appendix/}
\fi

%Check Ordner for Sourcecode
\ifdefined\ordnersource
\renewcommand{\ordnersource}{\bordner source/}
\else
\newcommand{\ordnersource}{\bordner source/}
\fi

