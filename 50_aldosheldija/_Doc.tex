\chapter{\docname}
\label{\docname}
\section{Allgemeine Beschreibung}

Ein sehr wichtiger Teil dieser Diplomarbeit ist auch die Admin Webseite. In dieser Webseite hat der Admin die M\"{o}glichkeit, alle Anzeigen zu verwalten und denen mit unterschiedlichen Inhalten zu bef\"{u}llen. Durch diese Webseite wird die Verwaltung einfacher und \"{u}bersichtlicher gemacht. Dort wird die Gelegenheit zu dem Admin gegeben, Anzeige hinzuf\"{u}gen, \"{a}ndern und l\"{o}schen, verschiedenen Layouts f\"{u}r unterschiedliche Anzeigen anzuordnen, Bilder zu verwalten und Benutzer zu administrieren. Ein anderes Feature zu dieser Webseite ist auch der Supplierplan. Hier kann der Stundeplan f\"{u}r bestimmte Tagen gemacht werden, damit er in die Anzeige dargestellte werden kann. Es wurde so gemacht, dass der Admin nur in ein paar Klicks das erledigen kann. Da unser Diploma nicht nur aus dieser Webseite besteht, wird unsere Webseite eine Schnittstelle sein, um die API-Key von Wetterbericht, URL der Webseite, von dem die letzten Posts gespeichert werden, und Chatbot Token zu verwalten.

Diese Webseite wurde so konzipiert, dass der Benutzer, der das verwenden wird, die Informationen und die Bedienung dieser Webseite sehr leicht und ohne Probleme folgt. Die Website ist auch in mehrere Sprachen angebetet, so dass noch leichter wird.

Wie gesehen, der Inhalt in dieser Webseite \"{a}ndert st\"{a}ndig und deshalb ist unsere Webseite, eine dynamische Website, die mit PHP und MySQL gemacht wird. F\"{u}r die Darstellung und das Design wurde Bootstrap verwendet. Um die Tabellen am besten darzustellen wurde auch Datables, ein JavaScript Plugin, verwendet. Regex wurde verwendet um die Validierung von unterschiedlichen User Input zu \"{u}berpr\"{u}fen und sicherstellen, dass der Input richtig ist.
 

Damit nicht jeder diesen Inhalten \"{a}ndern und l\"{o}schen kann, ist diese Webseite auch mit einen Login Bereich gesch\"{u}tzt. Nur bestimmte Personen, abh\"{a}ngig von welchem Rechten sie haben k\"{o}nnen dann die zugeh\"{o}rigen \"{A}nderungen machen. Falls ein Benutzer das Passwort vergisst, hat er die M\"{o}glichkeit das Passwort zur\"{u}ckzusetzen und wartet bis der Admin das best\"{a}tigt hat.


Der Zugriff auf dieser Webseite ist auf jedes Ger\"{a}t und von jedem Ort aus m\"{o}glich. Um diese Webseite anzuschauen ist nur ein Internet Verbindung notwendig.




\section{Technologien}

Dieses Kapitel setzt sich mit der Beschreibung der Technologien, die f\"{u}r die Erstellung des Admin Webseite verwenden wurden auseinander. 


\subsection{Bootstrap}


\subsection{PHP}



\subsection{MySQL}



\subsection{JavaScript}



\subsection{Visual Studio Code}


\subsection{Datatables}

\section{Rechte in die Webseite}

\section{Basis Seite}


\section{Displays}


\section{Layouts}

\section{Webseite auf mehrere Sprachen}
