\chapter{\docname}
\label{\docname}
In diesem Kapitel wird die Vorgehensweise der Technische L\"{o}sung bei der Admin Webseite und Client erkl\"{a}rt. Weiter werden auch die verwendeten Technologien aufgelistet. 

\section{Allgemeine Beschreibung}

\subsection{Admin Webseite}
Ein sehr wichtiger Teil dieser Diplomarbeit ist auch die Admin Webseite. In dieser Webseite hat der Admin die M\"{o}glichkeit, alle Anzeigen zu verwalten und denen mit unterschiedlichen Inhalten zu bef\"{u}llen. Durch diese Webseite wird die Verwaltung einfacher und \"{u}bersichtlicher gemacht. Dort wird die Gelegenheit zu dem Admin gegeben, Anzeige hinzuf\"{u}gen, \"{a}ndern und l\"{o}schen, verschiedenen Layouts f\"{u}r unterschiedliche Anzeigen anzuordnen, Bilder zu verwalten und Benutzer zu administrieren. Ein anderes Feature zu dieser Webseite ist auch der Supplierplan. Hier kann der Stundeplan f\"{u}r bestimmte Tagen gemacht werden, damit er in die Anzeige dargestellte werden kann. Es wurde so gemacht, dass der Admin nur in ein paar Klicks das erledigen kann. Da unser Diploma nicht nur aus dieser Webseite besteht, wird unsere Webseite eine Schnittstelle sein, um die API-Key von Wetterbericht, URL der Webseite, von dem die letzten Posts gespeichert werden, und Chatbot Token zu verwalten.

Diese Webseite wurde so konzipiert, dass der Benutzer, der das verwenden wird, die Informationen und die Bedienung dieser Webseite sehr leicht und ohne Probleme folgt. Die Website ist auch in mehrere Sprachen angebetet, so dass noch leichter wird.

Wie gesehen, der Inhalt in dieser Webseite \"{a}ndert st\"{a}ndig und deshalb ist unsere Webseite, eine dynamische Website, die mit PHP und MySQL gemacht wird. F\"{u}r die Darstellung und das Design wurde Bootstrap verwendet. Um die Tabellen am besten darzustellen wurde auch Datables, ein JavaScript Plugin, verwendet. Regex wurde verwendet um die Validierung von unterschiedlichen User Input zu \"{u}berpr\"{u}fen und sicherstellen, dass der Input richtig ist.
 

Damit nicht jeder diesen Inhalten \"{a}ndern und l\"{o}schen kann, ist diese Webseite auch mit einen Login Bereich gesch\"{u}tzt. Nur bestimmte Personen, abh\"{a}ngig von welchem Rechten sie haben k\"{o}nnen dann die zugeh\"{o}rigen \"{A}nderungen machen. Falls ein Benutzer das Passwort vergisst, hat er die M\"{o}glichkeit das Passwort zur\"{u}ckzusetzen und wartet bis der Admin das best\"{a}tigt hat.


Der Zugriff auf dieser Webseite ist auf jedes Ger\"{a}t und von jedem Ort aus m\"{o}glich. Um diese Webseite anzuschauen ist nur ein Internet Verbindung notwendig.

\subsection{Client}

Zu diese Diplomarbeit geh\"{o}rt auch der Client dazu. Als Client wird bezeichnet die Anzeige, in dem die Informationen dargestellt werden. Der Client besteht aus einem RaspberryPi, der mit Internet verbindet wird (WLAN\footnote{Wireless LAN} oder LAN\footnote{Local Area Network} Verbindung), und mit dem Server kommunizieren wird. Der RaspberryPi hat einen HDMI\footnote{High Definition Multimedia Interface} Anschluss, der mit einem Bildschirm verbindet wird. Um das System mehr kompatibel und umfassend, werden unterschiedliche Bildschirmgr\"{o}\ss{}en unterst\"{u}tzt. Der Anzeige wird automatisch registriert in der Datenbank mithilfe von einem Skript, der die Informationen beim Hochfahren des Ger\"{a}ts zu Datenbank schickt. Diese Anzeige wird beim Hochfahren des Betriebssystems die Webseite von der Anzeige \"{o}ffnen und das Anzeigen lassen. Diese Webseite wird im Vollbild ge\"{o}ffnet. Zus\"{a}tzlich werden die Bildschirmaustastung, Bildschirmschoner und Energieverwaltungssystem ausgeschaltet, damit die Anzeige immer eingeschaltet bleibt. Damit das System nicht abh\"{a}ngig von dem Internet ist, werden die Informationen f\"{u}r diese Anzeige von der Datenbank, die im Server liegt, kopiert und auf der lokalen Datenbank gespeichert. 

In dem Client laufen eine Apache Server, MySQL Datenbank und PHP, die f\"{u}r die Anzeige notwendig sind um die Informationen anzuzeigen. Falls der Anzeige noch nicht von der Admin freigegeben wurde, wird die Anzeige zu eine Welcome Seite umgeleitet, in dem nur die MAC-Adresse der Anzeige dargestellt wird.



\section{Technologien}

Dieses Kapitel setzt sich mit der Beschreibung der Technologien, die f\"{u}r die Erstellung des Admin Webseite und der Client verwenden wurden auseinander. 


\subsection{Admin Webseite}
\textbf{Bootstrap} (Abb. \ref{fi:bootstrap})
ist eine Open-Source und Kostenloses CSS\footnote{Cascading Style Sheet} Framework, die f\"{u}r die Herstellung von unterschiedlichen responsiven Webseiten hilft. Dieses Framework ist nicht nur aus CSS basiert, sondern die enth\"{a}lt auch JavaScript, die die Interaktion von dynamischer Webseite erh\"{o}ht. Bootstrap stellt fertige CSS Klassen zur Verf\"{u}gung, die f\"{u}r unterschiedliche Bildschirmgr\"{o}\ss{}en automatisch angepasst werden. Das wird durch ein Grid System erm\"{o}glicht.

\begin{figure}[h!]
	\centering
	\includegraphics[width=0.3\textwidth]{\ordnerfigures bootstrap.png}
	\caption{Bootstrap Logo}
	\label{fi:bootstrap}
\end{figure}

\textbf{PHP} (Abb. \ref{fi:php})
stand f\"{u}r Personal Home Page Tools und jetzt hei\ss{}t PHP Hypertext Preprocessor. Die aktuellste Version von PHP ist 7.4.0, die im Dezember 2019 ver\"{o}ffentlicht. PHP ist eine Serverseitige Programmiersprache, die f\"{u}r die Erstellung von dynamischen Webseiten dient. Mit Hilfe von PDO wird die Verbindung mit der Datenbank gemacht. PHP ist fast von alle Webhosting Services verf\"{u}gbar und unterst\"{u}tzt.

\begin{figure}[h!]
	\centering
	\includegraphics[width=0.3\textwidth]{\ordnerfigures php.png}
	\caption{PHP Logo}
	\label{fi:php}
\end{figure}



\textbf{JavaScript} 
(JS) ist eine kompakte, interpretierte oder just-in-time kompilierte Programmiersprache mit erstklassigen Funktionen. W\"{a}hrend es als Skriptsprache f\"{u}r Webseiten am bekanntesten ist, wird es auch von vielen anderen Umgebungen als Browsern verwendet, z. B. Node.js, Apache CouchDB und Adobe Acrobat. JavaScript ist eine prototypbasierte, auf mehreren Paradigmen basierende, dynamische Sprache mit einem Thread, die objektorientierte, imperative und deklarative Stile (z. B. funktionale Programmierung) unterst\"{u}tzt. Lesen Sie mehr \"{u}ber JavaScript.
Verwechseln Sie JavaScript nicht mit der Programmiersprache Java. Sowohl "Java" als auch "JavaScript" sind Marken oder eingetragene Marken von Oracle in den USA und anderen L\"{a}ndern. Die beiden Programmiersprachen haben jedoch eine sehr unterschiedliche Syntax, Semantik und Verwendung.

\begin{figure}[h!]
	\centering
	\includegraphics[width=0.3\textwidth]{\ordnerfigures js.jpg}
	\caption{JavaScript Logo}
	\label{fi:js}
\end{figure}

%https://developer.mozilla.org/en-US/docs/Web/JavaScripts


\textbf{Datatables}


\subsection{Client}

\textbf{MySQL}\footnote{My Structured Query Language}

F\"{u}r die Relationale Datenbank Management System wurde MySQL (Abb. \ref{fi:mysql}) gew\"{a}hlt. Dieses Datenbankverwaltungssystem wird gerne f\"{u}r kleine Projekte verwendet, weil es Open-Source und kostenlos ist. MySQL ist eine Software um Datenbank zu erstellen und zu verwalten. MySQL erm\"{o}glicht kontrollierten Zugriff auf die Datenbank. Die Daten werden auf Tabellen Struktur gespeichert, die in Dateien auf der Festplatte gespeichert sind. Diese Software ist kompatibel auf mehrere unterschiedlichen Plattformen wie z.B. Windows, Linux, MacOS usw.

\begin{figure}[h!]
	\centering
	\includegraphics[width=0.3\textwidth]{\ordnerfigures mysql.png}
	\caption{MySQL Logo}
	\label{fi:mysql}
\end{figure}

´´Das \textbf{Apache} Apache HTTP Server-Projekt ist ein Versuch, einen Open-Source-HTTP-Server f\"{u}r moderne Betriebssysteme wie UNIX und Windows zu entwickeln und zu warten. Ziel dieses Projekts ist es, einen sicheren, effizienten und erweiterbaren Server bereitzustellen, der HTTP-Dienste synchron mit den aktuellen HTTP-Standards bereitstellt. Dieser Server verwendet die Port 80 f\"{u}r HTTP und Port 443 f\"{u}r HTTPS. Die erste Version wurde im Jahr 1995 und der letzte stabile Version ist 2.4.41, der im August 2019 ver\"{o}ffentlicht wurde.``\cite[Seite 4]{test}

%https://httpd.apache.org/

\begin{figure}[h!]
	\centering
	\includegraphics[width=0.3\textwidth]{\ordnerfigures apache.png}
	\caption{Apache HTTP Server Logo}
	\label{fi:apache}
\end{figure}

\section{Technische L\"osung}
\subsection{Basis Seite}



\subsection{Displays}



\subsection{Layouts}



\subsection{Rechte in die Webseite}



\subsection{Webseite auf mehrere Sprachen}


