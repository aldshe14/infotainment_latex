\chapter{Umsetzung - Irena Bala}
\section{Allgemeine Beschreibungen}
In diesem Unterkapitel werden die verwendeten Technologien und Software-Ressourcen beschrieben.  Es werden die grundlegenden Theorien, die hinter diesen Technologien stehen, im Detail erläutert. Dazu werden auch die Gründe für die Auswahl der Software-Ressourcen erklärt. 
Die unterliegende Tabelle listet die verwendeten Technologien auf und daneben steht auch eine kurze Beschreibung für jede Technologie. \cite[Seite5]{einstein,knuthwebsite}
\subsubsection{Was ist Apache HTTP Server?} 
Der Apache HTTP\footnote{Hyper....}\nomenclature{HTTP}{hyper...} Server ist ein weltweit verbreitender Web Server. Dieser Server ist Open Source, das bedeutet, dass es keine Lizenz gekauft werden soll, um es zu verwenden. Es ist kompatibel auf allen kohärenteren Betriebssystemen, beispielsweise Linux, Windows, Mac OS und andere. Es bietet viele Versionen an, die zu unterschiedlichen Anwendungsgebiete passen und verbesserte Eigenschaften bereitstellen. Durch dieses Webservers können Webseiten erstellt werden. Die Erstellung der Webseiten erfolgt über serverseitige Scriptsprachen, die von dem Server selbst nicht unterstützt werden. Sie werden als Zusatzfunktionen angehängt. 
\\
\\
\textbf{Funktionsweise von Apache HTTP Server} \\
\\
„Obwohl Apache als Webserver bezeichnet wird, handelt es sich nicht um einen physischen Server. Apache ist eine Software, die auf einem Server ausgeführt wird. Beim Apache-Webserver handelt es sich um eine plattformübergreifende Software, die sowohl auf Unix/Linux wie auch auf Windows Servern ausgeführt werden kann. Seine Aufgabe ist es, eine Verbindung zwischen einem physischen Server mit den gespeicherten Webseiten und den Browsern der Internetuser herzustellen.
Wenn ein User eine URL in seinen Webbrowser eingibt, sendet der Browser eine HTTP oder HTTPS Anforderung an den Server, auf dem die Webseite gespeichert ist. Der auf dem Server installierte Apache Server verarbeitet die HTTP oder HTTPS Anforderung und gibt die angeforderten Webseiten zurück.“ \\
\\
Der Apache HTTP Server bietet viele Funktionalitäten an, die seine Entwicklungsumgebung vergrößern. Die wichtigste davon ist die Möglichkeit der Integration eines SSL-Zertifikats. Das ermöglicht die Übertragung der Daten in einer verschlüsselten Form. Die detaillierte Funktionsweise eines SSL Zertifikats wird in den Unterkapiteln beschrieben. 
\\ 
\\
\textbf{Was ist MySQL?} \\
\\
MySQL ist ein weitverbreitetes relationales Datenbanksystem. Die Datenbanksysteme werden allgemein zur Datenspeicherung und Datenverwaltung verwendet. Ein wichtiges Kriterium für die Datenspeicherung ist die Performanz. Diese Anforderung wird durch MySQL optimal erfüllt. Das ist auch der Grund, warum dieses Datenbanksystem so populär und bekannt ist. Die von MySQL für die Abarbeitung, Verwaltung und Systematisierung von Daten verwendete Sprache ist SQL. MySQL ist auch eine Open Source Software,
die in meisten Fällen in Verbindung mit serverseitigen Scriptsprachen wie PHP, vorkommt.
\\
\\
\textbf{Funktionsweise von MySQL} \\
\\
Das MySQL Datenbanksystem wird sehr häufig implementiert. Es gibt viele Unternehmen und Institutionen, die ihre Daten über eine gewisse Zeit speichern wollen. Das MySQL Datenbanksystem, das die Daten beinhaltet, wird als ein Server vorgesehen. Jeder, der versucht, Zugriff auf diese Daten zu haben, wird als ein Client vorgesehen. Der Server kann die erforderliche Zugänglichkeit erlauben oder nicht. Das hängt von den Clientrechten ab. Die Daten können von den Clients selektiert, bearbeitet oder gelöscht werden. Diese Ereignisse erfolgen durch SQL-Abfragen. Die SQL-Abfragen werden mithilfe der SQL Datenbanksprache erstellt. 
\\
\\ 
\textbf{Was ist PHP?} \\
\\
PHP ist eine serverseitige Programmiersprache. Das bedeutet, dass diese Sprache, um 
die vom Server auszuführenden Ereignissen zu programmieren, verwendet wird. 
PHP ist eine sehr verbreitete Programmiersprache, die am meisten zur Erstellung und Programmierung von Webseiten verwendet wird. Eigentlich ist PHP sehr flexibel, denn es einen großen Schnittstellenansatz anbietet.  Diese Programmiersprache kann auch im Zusammenhang mit Datenbanken genutzt werden. 
\\
\\
\textbf{Funktionsweise von PHP} \\
\\
Hier kommt das Client-Server Prinzip wieder vor. Der Webbrowser ist der Client und der Webserver ist der Server. Der mit PHP programmiertes Skript wird zum Webserver geschickt, danach erfolgt die Rückgabe einer HTML-Datei als Antwort zum Webbrowser, der in diesem Fall als Client betrachtet wird.
\\
\\
\textbf{Was ist JavaScript?} \\
\\
JavaScript ist eine Programmiersprache, die am meisten zur Erstellung von dynamischen Funktionalitäten bei Webseiten, verwendet wird. Die JavaScript Programmiersprache hat in der Vergangenheit nur eine beschränkte Anzahl von Funktionen angeboten, aber heutzutage bietet sie eine Vielzahl von Einsatzmöglichkeiten. 
\\
\\
\textbf{Mögliche Funktionen von JavaScript} \\
\\
„JavaScript wurde entwickelt, um dynamische HTML-Seiten per Webbrowser anzuzeigen. Die Verarbeitung von JavaScript erfolgt meist clientseitig direkt durch den Webbrowser.
Mit Hilfe der Skriptsprache JavaScript lassen sich viele dynamische Funktionen realisieren. Hier sind einige Beispiele für die Verwendung von JavaScript:
\begin{itemize}
	\item  dynamische Veränderung von Webseiten – zum Beispiel für die Anzeige eines formatierten und aktualisierten Datums
\end{itemize}
\begin{itemize}
	\item Prüfung von in Formularen eingegebenen Daten auf Plausibilität
\end{itemize}
\begin{itemize}
	\item Anzeige von Laufschriften oder Bannern
\end{itemize}
\begin{itemize}
	\item 	Öffnen und Anzeigen von Dialogfenstern
\end{itemize}
\begin{itemize}
	\item Aktualisieren von Daten einer Webseite ohne neu laden im Browser
\end{itemize}
\begin{itemize}
	\item 	Unterstützung der Eingabe von Daten durch den User
\end{itemize}
\begin{itemize}
	\item Veränderung von Texten oder Grafiken durch den Mauszeiger"
\end{itemize}
\textbf{Was ist Python?} \\
\\
Python ist eine objektorientierte Programmiersprache, aber kann auch in prozedurale Programmierung verwendet werden. Sie wurde ausschließlich zum Zweck der einfach einprägsamen Syntax entwickelt. Andererseits haben die Entwickler der Systematisierung des Codes große Bedeutung beigemessen. Wegen dieser angewandten Eigenschaften kann Python in die Gruppe der leichten Programmiersprachen aufgenommen werden. Diese Programmiersprache wird viel verwendet, aber was die Anwendungsumgebung besonders erhöht, ist die Möglichkeit andere Module anzuhängen. Es ist auch eine Open Source Software, der von den Programmierern verwendet, verändert, angepasst bzw. bearbeitet werden kann. Es wird meistens für komplexe Aufgaben verwendet werden, deswegen wird es als eine Hochsprache betrachtet.
\\
\\
\textbf{Merkmale von Python} 
\begin{itemize}
	\item Einfach einprägsame Syntax
\end{itemize}
\begin{itemize}
	\item Objektorientierte und prozedurale Programmiersprache
\end{itemize}
\begin{itemize}
	\item Open Source 
\end{itemize}
\begin{itemize}
	\item Hoches Niveau Programmiersprache
\end{itemize}
\begin{itemize}
	\item Leicht veränderbare Programmiersprache
\end{itemize}
\textbf{Was ist HTML?} \\
\\
HTML ist keine Programmiersprache, die wird für die Erstellung von Inhalten bei Webseiten verwendet. Diese Inhalte können Texte, Bilder oder andere Komponente sein. HTML wird als eine Auszeichnungssprache betrachtet. Sie ist nicht nur für die Erstellung von Webseite-Inhalten zuständig, sondern auch für ihr Design. Diese Sprache liegt mithilfe von bestimmten Tags die Struktur einer Webseite fest. Im Tag werden die Inhalte gespeichert. Es gibt bestimmte Tags für verschiedene Layout-Elemente. %\cite{einstein}
\\
\\
\textbf{Was ist CSS?} \\
\\
CSS wird im Zusammenhang mit HTML verwendet. Diese Methode wird unten genauer betrachtet. \\
„CSS steht für Cascading-Style-Sheets und ist eine Möglichkeit für HTML-Dokumente, den Inhalt einer Seite von den Designanweisungen der einzelnen Elemente, wie zum Beispiel Überschriften, Zitaten) zu entkoppeln.
Man kann eine CSS-Datei für eine Domain erstellen und diese auf allen Unterseiten als externe Ressource einbinden. Dies kann viel Zeit sparen, wenn feste Vorgaben für meine Designelemente vorhanden sind, die sich zwischen Dokumenten nicht ändern.“ 
\\
\\
\textbf{Was ist Raspberry PI?} \\
\\
Der Raspberry Pi ist ein kleiner Computer, dass nicht sehr viel kostet. Es bietet viele Möglichkeiten, um damit Programmierung und Scripting zu lernen. Dieser Computer ist für Schüler geeignet. Es braucht eine Tastatur, eine Maus, ein Netzteil, VGA und HDMI-VGA Konverter damit wir es verwenden können. 
„Der Raspberry PI ist ein Mini-Rechner, der für verschiedene Funktionalitäten geeignet ist. Mit einem Raspberry PI kann man verschiedene Sachen machen wie z.B: Linux Programmierung, physisches Rechnen, Shell Scripting usw. Ein VGA Kabel, ein HDMI Konverter und ein Netzteil sind erforderlich, um den Raspberry PI sofort in Betrieb zu nehmen. Das Betriebssystem kann nur mit einer SD-Karte gestartet werden. Die SD-Karte ist der wichtigste Teil von dem Raspberry PI.“

\section{Test}
\label{\docname}
