\chapter{Umsetzung - Irena Bala}
\section{Allgemeine Beschreibungen}
Bei dieser Diplomarbeit wurde darauf abgezielt, ein intelligentes System zu entwickeln, dass die täglichen Aufgaben des Menschen erleichtert und so viel wie möglich automatisch gesteuert wird. Um dieses Ziel zu erreichen wurden bestimmte Komponenten im System implementiert. Diese wurden mit Absicht so ausgewählt, um die zukünftige Erweiterung und Umsetzung des Projekts auf vielen Anwendungsgebiete zu erlauben.  Ein weiterer Zweck besteht also darin, das System zu vervollständigen und sein Entwicklungsgemeinde zu vergrößern. \\
Diese Komponente wurden in unterschiedlichen Bereichen unterteilt. Der erste Bereich ist die Datenbank. Die Datenbank ist der wesentliche Bestandteil des Systems, weil die Basis für die Speicherung der benötigten Daten bildet. In die Datenbank wurden alle von den verwendeten APIs gekommene Daten gespeichert. Außerdem wurden dort auch die Schuldaten gelegen. Diese Daten sind ein separater Teil des Projekts, die in den folgenden Kapiteln genauer erklärt werden. \\
Die Darstellung an Bildschirm von den gespeicherten Daten wurde durch eine Admin-Webseite realisiert. Es wurden viele Layouts für die Anzeige (Bildschirm) entworfen, damit die Informationen auf unterschiedlichen Weisen dargestellt werden. \\
Das Infotainment System funktioniert wie die meisten Systeme nach dem Server- Client Prinzip. Sowohl der Server, als auch der Client werden näher betrachtet, denn sie die Hauptbereiche des ganzen Systems sind. 
Ein weiterer interessanter Bereich ist der Chatbot. Chatbot wurde deswegen implementiert, weil es die Interaktion des Menschen mit dem System ermöglicht und dadurch wurde angenommen, dass die Umsetzung dieses Komponentes das Interesse des Menschen an dem System erhöhen wird. \\
Dies war eine allgemeine Beschreibung von den bis jetzige erreichte Ergebnisse. Entsprechend der jeweiligen Arbeitsaufteilung werden in den folgenden Kapiteln einige von den oben genannten Aspekten näher erläutert. \\
\subsection{Chatbot} 
In diesem Unterkapitel wird eine Einführung in Chatbot gemacht und dessen Umsetzung erklärt.
\subsubsection{Einführung in Chatbot}
Chatbot ist ein sehr wichtiger Komponent von der künstlichen Intelligenz. Es bietet eine Kommunikationsschnittstelle zwischen Menschen und technischen Systeme. Chatbot empfängt Anweisungen in Textform von den Menschen und überträgt diese so, dass die Systeme diesen Anweisungen entsprechen. Basierend auf was Chatbots anbieten, werden als sehr schlaue Komponente angesehen, die immer mehr implementiert werden. 
\subsubsection{Umsetzung von Chatbot}
Chatbot wurde bei dieser Diplomarbeit so implementiert, dass es den Schülern die Möglichkeit gibt, aufgenommene Bilder zum Chatbot zu schicken und dann diese Bilder werden automatisch auf dem Bildschirm angezeigt. Das ist auch die wesentliche Funktionalität von Chatbot bei diesem Projekt. \\
Um den Chatbot zu implementieren, haben viele kleine Prozesse stattgefunden, die als weitere oder zusätzliche Funktionen angesehen werden können.\\
Zuerst wurde Telegram Bot API als eine Schnittstelle für die Chatbot-Implementierung ausgewählt. Durch diese API können neue Bots erstellt, bearbeitet und verändert werden. Telegram Bot API funktioniert gleich wie die anderen Kommunikationsapplikation z.B Whatsapp. Der wesentliche Unterschied ist, dass bei dieser Applikation wird nicht nur die Möglichkeit mit anderen Menschen zu chatten angeboten, sondern auch mit Chatbots. Jedes Bot, das erstellt wird, bekommt ein Token, dass eindeutig für das Bot ist, wie die Telefonnummer für uns eindeutig ist. \\
Die komplette Funktionalität des Bots wurde in RaspberryPI Server, mithilfe der Python Programmiersprache programmiert. \\
Je nachdem ob die Person, die mit dem Bot chatten will, als ein normaler Benutzer oder ein Administrator in der Datenbank definiert ist, werden ihm verschiedene Funktionen im Zusammenhang mit dem Bot zur Verfügung gestellt. Die Nachrichten, die zu dem Bot geschickt werden, werden nach Inhalt überprüft. Basierend auf Inhalt der Nachrichten, wird der Bot auf verschiedene Weisen reagieren.\\
Die Art der Umsetzung bzw. Realisierung aller diesen Funktionen kann im Unterkapitel 2.1 gelesen werden.   

\subsection{Server}
Das Infotainment System wie die anderen technischen Systeme, funktioniert nach dem Client-Server Prinzip. Das bedeutet, es gibt einen Server und einen Client, die miteinander kommunizieren. Der Client ist in diesem Projekt die Anzeige bzw. das Bildschirm, dass die von dem Server bekommene Daten darstellt. \\
Im Server liegen aber alle benötigten Informationen für die Darstellung. Diese Informationen sind auf der Datenbank gespeichert, die auf dem Server stattfindet.\\
Der Server beinhaltet auch die grundlegenden Skripts für die Chatbot-Implementierung und für die Programmierung der Admin-Webseite. Im Server wurde auch das SSL Zertifikat für eine sichere Datenübertragung erstellt.\\
Die Kommunikation zwischen dem Client und dem Server wird dann aufgebaut, wenn Daten von dem Server ausgewählt und zum Client geschickt werden. 
Der Server ist der grundlegende Teil des Projekts. Es enthält alle benötigten Ressourcen für die vollständige Umsetzung des Systems.
\subsection{Technologien}
In diesem Unterkapitel werden die verwendeten Technologien und Software-Ressourcen beschrieben.  Es werden die grundlegenden Theorien, die hinter diesen Technologien stehen, im Detail erläutert. Dazu werden auch die Gründe für die Auswahl der Software-Ressourcen erklärt. 
Die unterliegende Tabelle listet die verwendeten Technologien auf und daneben steht auch eine kurze Beschreibung für jede Technologie. \cite[Seite5]{einstein,knuthwebsite}
\begin{table}[h]
	\begin{center}
 \caption{Technologien}
\label{tab:Tabelle1}
\resizebox{\textwidth}{!}{
	\begin{tabular}{ | l |l |p{5cm} |} 
		\hline
		\textbf{Name} & \textbf{Beschreibung}\\
		\hline
		Apache HTTP Server & Webserver   \\ \hline
	    MySQL& Relationales Datenbanksystem\\ \hline
		PHP	& Serverseitige Programmiersprache\\ \hline
		JavaScript & Programmiersprache zur dynamischen 
		Veränderung von Webseiten \\ \hline 
		Python	& Objektorientierte/ prozedurale Programmiersprache \\ \hline
		HTML &	Auszeichnungssprache zur Erstellung von Inhalten bei Webseiten \\ \hline
		CSS	& Methode, zur Entkopplung von Designanweisungen einer HTML Datei \\ \hline
		Wetter API &	Schnittstelle zur Aufnahme von Wetterdaten aus großen Wettervorhersage-Datenbanken \\ \hline
		JSON	& strukturiertes Dateiformat \\ \hline 
		Telegram API &	Schnittstelle zur Implementierung von Chatbot \\ \hline
		Raspberry PI &	Minirechner, der für Scripting, Linux Programmierung geeignet ist \\ \hline
		SSL &	Methode zur verschlüsselten Datenübertragung zwischen Browser und Server \\ \hline
		
\end{tabular} }
%\end{adjustbox} 
\end{center}
\end{table} 
\subsubsection{Was ist Apache HTTP Server?} 
Der Apache HTTP\footnote{Hyper....}\nomenclature{HTTP}{hyper...} Server ist ein weltweit verbreitender Web Server. Dieser Server ist Open Source, das bedeutet, dass es keine Lizenz gekauft werden soll, um es zu verwenden. Es ist kompatibel auf allen kohärenteren Betriebssystemen, beispielsweise Linux, Windows, Mac OS und andere. Es bietet viele Versionen an, die zu unterschiedlichen Anwendungsgebiete passen und verbesserte Eigenschaften bereitstellen. Durch dieses Webservers können Webseiten erstellt werden. Die Erstellung der Webseiten erfolgt über serverseitige Scriptsprachen, die von dem Server selbst nicht unterstützt werden. Sie werden als Zusatzfunktionen angehängt. 
\subsubsection{Funktionsweise von Apache HTTP Server} 
„Obwohl Apache als Webserver bezeichnet wird, handelt es sich nicht um einen physischen Server. Apache ist eine Software, die auf einem Server ausgeführt wird. Seine Aufgabe ist es, eine Verbindung zwischen einem physischen Server mit den gespeicherten Webseiten und den Browsern der Internetuser herzustellen. \\
Wenn ein User eine URL in seinen Webbrowser eingibt, sendet der Browser eine HTTP oder HTTPS Anforderung an den Server, auf dem die Webseite gespeichert ist.“ \\
\\
Der Apache HTTP Server bietet viele Funktionalitäten an, die seine Entwicklungsumgebung vergrößern. Die wichtigste davon ist die Möglichkeit der Integration eines SSL-Zertifikats. Das ermöglicht die Übertragung der Daten in einer verschlüsselten Form. Die detaillierte Funktionsweise eines SSL Zertifikats wird in den Unterkapiteln beschrieben. 
\subsubsection{Was ist MySQL?} 
MySQL ist ein weitverbreitetes relationales Datenbanksystem. Die Datenbanksysteme werden allgemein zur Datenspeicherung und Datenverwaltung verwendet. Ein wichtiges Kriterium für die Datenspeicherung ist die Performanz. Diese Anforderung wird durch MySQL optimal erfüllt. Das ist auch der Grund, warum dieses Datenbanksystem so populär und bekannt ist. Die von MySQL für die Abarbeitung, Verwaltung und Systematisierung von Daten verwendete Sprache ist SQL. MySQL ist auch eine Open Source Software,
die in meisten Fällen in Verbindung mit serverseitigen Scriptsprachen wie PHP, vorkommt.
\subsubsection{Funktionsweise von MySQL} 
Das MySQL Datenbanksystem wird sehr häufig implementiert. Es gibt viele Unternehmen und Institutionen, die ihre Daten über eine gewisse Zeit speichern wollen. Das MySQL Datenbanksystem, das die Daten beinhaltet, wird als ein Server vorgesehen. Jeder, der versucht, Zugriff auf diese Daten zu haben, wird als ein Client vorgesehen. Der Server kann die erforderliche Zugänglichkeit erlauben oder nicht. Das hängt von den Clientrechten ab. Die Daten können von den Clients selektiert, bearbeitet oder gelöscht werden. Diese Ereignisse erfolgen durch SQL-Abfragen. Die SQL-Abfragen werden mithilfe der SQL Datenbanksprache erstellt. 
\subsubsection{Was ist PHP?} 
PHP ist eine serverseitige Programmiersprache. Das bedeutet, dass diese Sprache, um 
die vom Server auszuführenden Ereignissen zu programmieren, verwendet wird. 
PHP ist eine sehr verbreitete Programmiersprache, die am meisten zur Erstellung und Programmierung von Webseiten verwendet wird. Eigentlich ist PHP sehr flexibel, denn es einen großen Schnittstellenansatz anbietet.  Diese Programmiersprache kann auch im Zusammenhang mit Datenbanken genutzt werden. 
\subsubsection{Funktionsweise von PHP} 
Hier kommt das Client-Server Prinzip wieder vor. Der Webbrowser ist der Client und der Webserver ist der Server. Der mit PHP programmiertes Skript wird zum Webserver geschickt, danach erfolgt die Rückgabe einer HTML-Datei als Antwort zum Webbrowser, der in diesem Fall als Client betrachtet wird.
\subsubsection{Was ist JavaScript?} 
JavaScript ist eine Programmiersprache, die am meisten zur Erstellung von dynamischen Funktionalitäten bei Webseiten, verwendet wird. Die JavaScript Programmiersprache hat in der Vergangenheit nur eine beschränkte Anzahl von Funktionen angeboten, aber heutzutage bietet sie eine Vielzahl von Einsatzmöglichkeiten. 
\subsubsection{Mögliche Funktionen von JavaScript} 
„JavaScript wurde entwickelt, um dynamische HTML-Seiten per Webbrowser anzuzeigen. Die Verarbeitung von JavaScript erfolgt meist clientseitig direkt durch den Webbrowser. \\
Mit Hilfe der Skriptsprache JavaScript lassen sich viele dynamische Funktionen realisieren. Hier sind einige Beispiele für die Verwendung von JavaScript:
\begin{itemize}
	\item  dynamische Veränderung von Webseiten – zum Beispiel für die Anzeige eines formatierten und aktualisierten Datums
\end{itemize}
\begin{itemize}
	\item Prüfung von in Formularen eingegebenen Daten auf Plausibilität
\end{itemize}
\begin{itemize}
	\item Anzeige von Laufschriften oder Bannern
\end{itemize}
\begin{itemize}
	\item 	Öffnen und Anzeigen von Dialogfenstern
\end{itemize}
\begin{itemize}
	\item Aktualisieren von Daten einer Webseite ohne neu laden im Browser
\end{itemize}
\begin{itemize}
	\item 	Unterstützung der Eingabe von Daten durch den User
\end{itemize}
\begin{itemize}
	\item Veränderung von Texten oder Grafiken durch den Mauszeiger"
\end{itemize}
\subsubsection{Was ist Python?} 
Python ist eine objektorientierte Programmiersprache, aber kann auch in prozedurale Programmierung verwendet werden. Sie wurde ausschließlich zum Zweck der einfach einprägsamen Syntax entwickelt. Andererseits haben die Entwickler der Systematisierung des Codes große Bedeutung beigemessen. Wegen dieser angewandten Eigenschaften kann Python in die Gruppe der leichten Programmiersprachen aufgenommen werden. Diese Programmiersprache wird viel verwendet, aber was die Anwendungsumgebung besonders erhöht, ist die Möglichkeit andere Module anzuhängen. Es ist auch eine Open Source Software, der von den Programmierern verwendet, verändert, angepasst bzw. bearbeitet werden kann. Es wird meistens für komplexe Aufgaben verwendet werden, deswegen wird es als eine Hochsprache betrachtet.
\subsubsection{Merkmale von Python} 
\begin{itemize}
	\item Einfach einprägsame Syntax
\end{itemize}
\begin{itemize}
	\item Objektorientierte und prozedurale Programmiersprache
\end{itemize}
\begin{itemize}
	\item Open Source 
\end{itemize}
\begin{itemize}
	\item Hoches Niveau Programmiersprache
\end{itemize}
\begin{itemize}
	\item Leicht veränderbare Programmiersprache
\end{itemize}
\subsubsection{Was ist HTML?} 
HTML ist keine Programmiersprache, die wird für die Erstellung von Inhalten bei Webseiten verwendet. Diese Inhalte können Texte, Bilder oder andere Komponente sein. HTML wird als eine Auszeichnungssprache betrachtet. Sie ist nicht nur für die Erstellung von Webseite-Inhalten zuständig, sondern auch für ihr Design. Diese Sprache liegt mithilfe von bestimmten Tags die Struktur einer Webseite fest. Im Tag werden die Inhalte gespeichert. Es gibt bestimmte Tags für verschiedene Layout-Elemente. %\cite{einstein}
\subsubsection{Was ist CSS?} 
CSS wird im Zusammenhang mit HTML verwendet. Diese Methode wird unten genauer betrachtet. \\
„CSS steht für Cascading-Style-Sheets und ist eine Möglichkeit für HTML-Dokumente, den Inhalt einer Seite von den Designanweisungen der einzelnen Elemente, wie zum Beispiel Überschriften, Zitaten) zu entkoppeln.“ 
\subsubsection{Was ist Raspberry PI?} 
Der Raspberry PI ist ein Minirechner, die zur Linux Programmierung, Shell Scripting und Realisierung von technischen Projekten verwendet wird. Es braucht eine Tastatur, eine Maus, ein Netzteil, VGA und HDMI-VGA Konverter, damit es genutzt werden kann. Die Konfiguration von einem Raspberry PI erfolgt durch eine SD-Karte. Diese SD-Karte beinhaltet das Image, wo das Betriebssystem liegt. Ein Raspberry PI kann in Zusammenhang mit vielen anderen Komponenten verwendet werden. 
\subsubsection{Was ist SSL?} 
SSL steht für Secure Socket Layer und ist für die verschlüsselte Übertragung der Daten vom Browser zum Server verantwortlich. Die Verbindung zwischen dem Server und dem Browser erfolgt durch das HTTPS Protokoll. Das ist ein Kommunikationsprotokoll, das eine verschlüsselte Datenübertragung ermöglicht. Heutzutage wird TLS am meisten verwendet, der der neueste und modernste Standard von SSL ist.
\subsubsection{SSL-Verschlüsselung}
Um eine verschlüsselte Verbindung zwischen einem Browser und einem Server aufzubauen, werden SSL – Zertifikaten integriert. Mittels ein SSL Zertifikats wird die Authentizität einer Webseite überprüft. Das SSL Zertifikat wird von einer Zertifizierungsstelle, erzeugt. Diese Zertifizierungsstelle heißt CA und erfordert einige Daten von dem Antragsteller, die für die Erstellung des Zertifikats notwendig sind. Als nächstes, erzeugt der Antragsteller für die Entschlüsselung und Verschlüsselung der zwischenübertragenen Daten ein öffentlicher, -und ein privater Schlüssel. Je grösser die Lange des Schlüssels ist, desto sicherer und besser ist. Meistens werden Schlüssel mit einer Länge von 256 Bit verwendet.
\subsubsection{Was ist Telegram Bot API?} 
Telegram Bot API ist eine Schnittstelle, die die Implementierung von Chatbot ermöglicht. Es bietet verschiedene Funktionen an, nämlich die Einrichtung, Erstellung und die Verarbeitung von Bots. Diese Funktionen sind in der eigenen Dokumentation von Telegram Bot API klar beschrieben.  
\subsubsection{Was ist Wetter API?} 
„Wetter APIs sind Schnittstellen, die die Verbindung zu einer großen Wettervorhersage-Datenbank und die Aufnahme benötigter Daten ermöglichen.“ 
\subsubsection{Was ist JSON?} 
„JSON bietet einen einfachen Standard für die strukturierte Kodierung von Daten in Form von menschenlesbarem Text. Dies bietet Vorteile bei einer automatisierten Weiterverarbeitung, macht sie aber auch einer manuellen Inspektion und Überarbeitung besser zugänglich.“ \\
\\
In der untenstehenden Tabelle sind alle Technologien zusammen mit dem Bereich wo sie gehören ersichtlich. 
\begin{table}[h]
	\begin{center}
		\caption{Bereiche und Technologien}
		\label{tab:Tabelle1}
		\resizebox{\textwidth}{!}{
			\begin{tabular}{ | l |l |p{2cm} |} 
				\hline
				\textbf{Bereich} & \textbf{Technologie}\\ \hline 
				Datenbank &	Apache HTTP Server, MySQL, PHP \\ \hline
				Anzeige &	HTML, CSS, JavaScript \\ \hline
				Server &	RaspberryPI, SSL \\ \hline
				Wetterdaten	 &Wetter API, JSON \\ \hline
				Chatbot &	Telegram API, Python \\ \hline
				
				\hline
				 
				
		\end{tabular} }
	\end{center}
\end{table} 
\section{Technische Lösungen}
\subsubsection{Structured Software Design}
\subsubsection{Konfiguration von Raspberry PI Server}
\begin{itemize}
	\item Zuerst wurde ein Image daraufgespielt, die das Betriebssystem von Raspberry PI beinhaltet
\end{itemize}
\begin{itemize}
	\item Danach wurden die folgenden gebrauchten Paketen installiert: git, vim, apache2, python-pip, telepot, php php-mbstring, mariadb-server php-mysql, phpmyadmin
\end{itemize}
\begin{itemize}
	\item In der Konfigurationsdatei wurde die IP-Adresse von dem Raspberry PI angelegt
\end{itemize}
\begin{itemize}
	\item Der nächste Schritt war die Erstellung der Datenbank und der dazugehörigen Tabellen
\end{itemize}
\begin{itemize}
	\item Dann wurden die Benutzer angelegt und die Rechte vergeben
\end{itemize}
\begin{itemize}
	\item Als letztens wurde SSH aktiviert, damit eine sichere Verbindung zu diesem Server von einem externen Gerät ermöglicht werden kann
\end{itemize}
\subsubsection{Datenbank}
\begin{itemize}
	\item Zuerst wurde ein ERD Diagramm mit Papier gezeichnet. Das Ziel war die richtige Erstellung der benötigten Tabellen. Die Tabellen wurden mit den Spalten und ihren Datentypen erstellt. Es wurden auch die Kardinalitäten dazwischen gezeichnet. Die erstellten Tabellen waren:
	
	\begin{itemize}
		\item Supplierplan
	\end{itemize}
    \begin{itemize}
    	\item Stundenplan
    \end{itemize}
    \begin{itemize}
    	\item Wetterdaten
    \end{itemize}
    \begin{itemize}
	\item Chatbot Users
    \end{itemize}	
\end{itemize}
\begin{itemize}
	\item Basierend auf das ERD Diagramm wurden dann die Tabellen mit phpmyadmin erstellt
\end{itemize}
\begin{itemize}
	\item Dann sind für alle Tabellen mit MySQL Workbench die entsprechenden gespeicherten Prozeduren erstellt 
\end{itemize}	
\subsubsection{SSL Verschlüsselung}
Es wurde ein SSL Zertifikat für den Apache HTTP Server eingerichtet 
\begin{itemize}
	\item Zuerst wurde das SSL Modul für Apache aktiviert
\end{itemize}
\begin{itemize}
	\item Danach wurde das SSL Zertifikat erstellt. 
\end{itemize}
\begin{itemize}
	\item Nach der Erstellung des SSL Zertifikats werden einige Eingaben geschickt, die erfüllt werden sollen. 
\end{itemize}
\begin{itemize}
	\item Danach wurde die Datei /etc/apache2/sites-available/default-ssl.conf geöffnet.
\end{itemize}
\begin{itemize}
	\item Unter der Zeile, wo SSL Engine On steht, wurden die erstellten Zertifikatdateien zugefügt
\end{itemize}
\begin{itemize}
	\item Danach wurde der Virtuellhost mit SSL aktiviert. 
\end{itemize}
\begin{itemize}
	\item Es wurde ein System reboot gemacht und der Apache Server noch einmal gestartet
\end{itemize}
\subsubsection{Wetterdaten}
\begin{itemize}
	\item Als erster Schritt erfolgte die Registrierung bei openweathermap.org 
\end{itemize}
\begin{itemize}
	\item 	Danach wurde ein API Key bekannt gegeben, mit dem die API Call gemacht werden kann
\end{itemize}
\begin{itemize}
	\item Die Antwort von API Call wird im JSON-Format gegeben
\end{itemize}
\begin{itemize}
	\item Danach erfolgt die Speicherung der bestimmten Parameter, die von dem API CALL kommen, in der Datenbank
\end{itemize}
\begin{itemize}
	\item Als letztens werden mithilfe von PHP die in der Datenbank gespeicherten Parameter aufgerufen und auf dem Bildschirm angezeigt
\end{itemize}
\subsubsection{Anzeige}	

\label{\docname}
