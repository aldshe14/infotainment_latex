\chapter{\docname}
\label{\docname}
Am Ende wird dieses System in der Schule implementiert. Dieses System besteht aus universellen Komponenten und ist skalierbar. Das bedeutet, es kann leicht auch bei anderen Institutionen eingebaut werden, wo die entsprechende Informationen angezeigt werden. 

Das System kann in unterschiedlichen Art und Weisen erweitert werden. Es kann beispielweise die Funktionalit\"at hinzugef\"ugt werden, eine Interaktion der Benutzer mit dem Bildschirm zu haben. Es kann f\"ur die Erkennung der Benutzer eine Kamera verwendet werden und danach k\"onnen die Daten dieser Person auf dem Bildschirm angezeigt werden. Die Benutzer sollen selbst die Informationen, die sie anzeigen wollen, aussuchen. In der Schule k\"onnen z.B die Sch\"uler ihre Noten oder Fehlstunden anschauen. 

Eine andere Erweiterung des Systems kann die Anzeige von mehreren Informationen sein. Es k\"onnen auch Videos auf dem Bildschirm dargestellt, was das System noch innovativer macht. 

Die Interessenten f\"ur dieses System sind die Firmen bzw. die verschiedenen Unternehmen. Sie k\"onnen dieses System in ihren Geb\"auden umsetzen, damit die entsprechenden Informationen von der Institution angezeigt werden. 
Eigentlich, bietet das Infotainment System in dem jetztigen Zustand vielf\"altige M\"oglichkeiten. Es kann aber auch weiterentwickelt werden und mit mehreren Funktionen vervollst\"andigt werden.

