\chapter{\docname}
\label{\docname}
\section{Allgemeine Beschreibung}

\section{Technologien}


\subsection{Bootstrap}


\subsection{PHP}
PHP steht f\"{u}r Personal Home Page Tools". Das war im Jahr 1994, als PHP
von Rasmus Lerdorf entwickelt wurde. Der PHP Hypertext Preprocessor (PHP)
ist eine Programmiersprache. Es wird normalerweise verwendet, um dynamische
Webseiteninhalte oder dynamische Bilder zu erstellen, die auf Websites verwendet
werden.4 PHP ist als Skriptsprache bei fast allen Webhostern verf\"{u}gbar und es
existiert eine gro\"{y}e Entwicklergemeinde. Dies sorgt f\"{u}r eine maximale Flexibilit\"{a}t
in der Umsetzung unterschiedlichster Anforderungen. Zudem besteht keine Abh\"{a}ngigkeit von einzelnen Anbietern. Die eigentlichen PHP-Skripte bestehen aus
reinen Textdateien und lassen sich einfach anpassen und pegen



\subsection{MySQL}



\subsection{JavaScript}



\subsection{Visual Studio Code}


\subsection{Datatables}

\section{Rechte in die Webseite}

\section{Basis Seite}


\section{Displays}


\section{Layouts}

\section{Webseite auf mehrere Sprachen}
